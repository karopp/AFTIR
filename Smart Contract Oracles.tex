
%% bare_conf.tex
%% V1.4b
%% 2015/08/26
%% by Michael Shell
%% See:
%% http://www.michaelshell.org/
%% for current contact information.
%%
%% This is a skeleton file demonstrating the use of IEEEtran.cls
%% (requires IEEEtran.cls version 1.8b or later) with an IEEE
%% conference paper.
%%
%% Support sites:
%% http://www.michaelshell.org/tex/ieeetran/
%% http://www.ctan.org/pkg/ieeetran
%% and
%% http://www.ieee.org/

%%*************************************************************************
%% Legal Notice:
%% This code is offered as-is without any warranty either expressed or
%% implied; without even the implied warranty of MERCHANTABILITY or
%% FITNESS FOR A PARTICULAR PURPOSE! 
%% User assumes all risk.
%% In no event shall the IEEE or any contributor to this code be liable for
%% any damages or losses, including, but not limited to, incidental,
%% consequential, or any other damages, resulting from the use or misuse
%% of any information contained here.
%%
%% All comments are the opinions of their respective authors and are not
%% necessarily endorsed by the IEEE.
%%
%% This work is distributed under the LaTeX Project Public License (LPPL)
%% ( http://www.latex-project.org/ ) version 1.3, and may be freely used,
%% distributed and modified. A copy of the LPPL, version 1.3, is included
%% in the base LaTeX documentation of all distributions of LaTeX released
%% 2003/12/01 or later.
%% Retain all contribution notices and credits.
%% ** Modified files should be clearly indicated as such, including  **
%% ** renaming them and changing author support contact information. **
%%*************************************************************************


% *** Authors should verify (and, if needed, correct) their LaTeX system  ***
% *** with the testflow diagnostic prior to trusting their LaTeX platform ***
% *** with production work. The IEEE's font choices and paper sizes can   ***
% *** trigger bugs that do not appear when using other class files.       ***                          ***
% The testflow support page is at:
% http://www.michaelshell.org/tex/testflow/



\documentclass[conference]{IEEEtran}
% Some Computer Society conferences also require the compsoc mode option,
% but others use the standard conference format.
%
% If IEEEtran.cls has not been installed into the LaTeX system files,
% manually specify the path to it like:
% \documentclass[conference]{../sty/IEEEtran}





% Some very useful LaTeX packages include:
% (uncomment the ones you want to load)


% *** MISC UTILITY PACKAGES ***
%
%\usepackage{ifpdf}
% Heiko Oberdiek's ifpdf.sty is very useful if you need conditional
% compilation based on whether the output is pdf or dvi.
% usage:
% \ifpdf
%   % pdf code
% \else
%   % dvi code
% \fi
% The latest version of ifpdf.sty can be obtained from:
% http://www.ctan.org/pkg/ifpdf
% Also, note that IEEEtran.cls V1.7 and later provides a builtin
% \ifCLASSINFOpdf conditional that works the same way.
% When switching from latex to pdflatex and vice-versa, the compiler may
% have to be run twice to clear warning/error messages.






% *** CITATION PACKAGES ***
%
%\usepackage{cite}
% cite.sty was written by Donald Arseneau
% V1.6 and later of IEEEtran pre-defines the format of the cite.sty package
% \cite{} output to follow that of the IEEE. Loading the cite package will
% result in citation numbers being automatically sorted and properly
% "compressed/ranged". e.g., [1], [9], [2], [7], [5], [6] without using
% cite.sty will become [1], [2], [5]--[7], [9] using cite.sty. cite.sty's
% \cite will automatically add leading space, if needed. Use cite.sty's
% noadjust option (cite.sty V3.8 and later) if you want to turn this off
% such as if a citation ever needs to be enclosed in parenthesis.
% cite.sty is already installed on most LaTeX systems. Be sure and use
% version 5.0 (2009-03-20) and later if using hyperref.sty.
% The latest version can be obtained at:
% http://www.ctan.org/pkg/cite
% The documentation is contained in the cite.sty file itself.






% *** GRAPHICS RELATED PACKAGES ***
%
\ifCLASSINFOpdf
  % \usepackage[pdftex]{graphicx}
  % declare the path(s) where your graphic files are
  % \graphicspath{{../pdf/}{../jpeg/}}
  % and their extensions so you won't have to specify these with
  % every instance of \includegraphics
  % \DeclareGraphicsExtensions{.pdf,.jpeg,.png}
\else
  % or other class option (dvipsone, dvipdf, if not using dvips). graphicx
  % will default to the driver specified in the system graphics.cfg if no
  % driver is specified.
  % \usepackage[dvips]{graphicx}
  % declare the path(s) where your graphic files are
  % \graphicspath{{../eps/}}
  % and their extensions so you won't have to specify these with
  % every instance of \includegraphics
  % \DeclareGraphicsExtensions{.eps}
\fi
% graphicx was written by David Carlisle and Sebastian Rahtz. It is
% required if you want graphics, photos, etc. graphicx.sty is already
% installed on most LaTeX systems. The latest version and documentation
% can be obtained at: 
% http://www.ctan.org/pkg/graphicx
% Another good source of documentation is "Using Imported Graphics in
% LaTeX2e" by Keith Reckdahl which can be found at:
% http://www.ctan.org/pkg/epslatex
%
% latex, and pdflatex in dvi mode, support graphics in encapsulated
% postscript (.eps) format. pdflatex in pdf mode supports graphics
% in .pdf, .jpeg, .png and .mps (metapost) formats. Users should ensure
% that all non-photo figures use a vector format (.eps, .pdf, .mps) and
% not a bitmapped formats (.jpeg, .png). The IEEE frowns on bitmapped formats
% which can result in "jaggedy"/blurry rendering of lines and letters as
% well as large increases in file sizes.
%
% You can find documentation about the pdfTeX application at:
% http://www.tug.org/applications/pdftex





% *** MATH PACKAGES ***
%
%\usepackage{amsmath}
% A popular package from the American Mathematical Society that provides
% many useful and powerful commands for dealing with mathematics.
%
% Note that the amsmath package sets \interdisplaylinepenalty to 10000
% thus preventing page breaks from occurring within multiline equations. Use:
%\interdisplaylinepenalty=2500
% after loading amsmath to restore such page breaks as IEEEtran.cls normally
% does. amsmath.sty is already installed on most LaTeX systems. The latest
% version and documentation can be obtained at:
% http://www.ctan.org/pkg/amsmath





% *** SPECIALIZED LIST PACKAGES ***
%
%\usepackage{algorithmic}
% algorithmic.sty was written by Peter Williams and Rogerio Brito.
% This package provides an algorithmic environment fo describing algorithms.
% You can use the algorithmic environment in-text or within a figure
% environment to provide for a floating algorithm. Do NOT use the algorithm
% floating environment provided by algorithm.sty (by the same authors) or
% algorithm2e.sty (by Christophe Fiorio) as the IEEE does not use dedicated
% algorithm float types and packages that provide these will not provide
% correct IEEE style captions. The latest version and documentation of
% algorithmic.sty can be obtained at:
% http://www.ctan.org/pkg/algorithms
% Also of interest may be the (relatively newer and more customizable)
% algorithmicx.sty package by Szasz Janos:
% http://www.ctan.org/pkg/algorithmicx




% *** ALIGNMENT PACKAGES ***
%
%\usepackage{array}
% Frank Mittelbach's and David Carlisle's array.sty patches and improves
% the standard LaTeX2e array and tabular environments to provide better
% appearance and additional user controls. As the default LaTeX2e table
% generation code is lacking to the point of almost being broken with
% respect to the quality of the end results, all users are strongly
% advised to use an enhanced (at the very least that provided by array.sty)
% set of table tools. array.sty is already installed on most systems. The
% latest version and documentation can be obtained at:
% http://www.ctan.org/pkg/array


% IEEEtran contains the IEEEeqnarray family of commands that can be used to
% generate multiline equations as well as matrices, tables, etc., of high
% quality.




% *** SUBFIGURE PACKAGES ***
%\ifCLASSOPTIONcompsoc
%  \usepackage[caption=false,font=normalsize,labelfont=sf,textfont=sf]{subfig}
%\else
%  \usepackage[caption=false,font=footnotesize]{subfig}
%\fi
% subfig.sty, written by Steven Douglas Cochran, is the modern replacement
% for subfigure.sty, the latter of which is no longer maintained and is
% incompatible with some LaTeX packages including fixltx2e. However,
% subfig.sty requires and automatically loads Axel Sommerfeldt's caption.sty
% which will override IEEEtran.cls' handling of captions and this will result
% in non-IEEE style figure/table captions. To prevent this problem, be sure
% and invoke subfig.sty's "caption=false" package option (available since
% subfig.sty version 1.3, 2005/06/28) as this is will preserve IEEEtran.cls
% handling of captions.
% Note that the Computer Society format requires a larger sans serif font
% than the serif footnote size font used in traditional IEEE formatting
% and thus the need to invoke different subfig.sty package options depending
% on whether compsoc mode has been enabled.
%
% The latest version and documentation of subfig.sty can be obtained at:
% http://www.ctan.org/pkg/subfig




% *** FLOAT PACKAGES ***
%
%\usepackage{fixltx2e}
% fixltx2e, the successor to the earlier fix2col.sty, was written by
% Frank Mittelbach and David Carlisle. This package corrects a few problems
% in the LaTeX2e kernel, the most notable of which is that in current
% LaTeX2e releases, the ordering of single and double column floats is not
% guaranteed to be preserved. Thus, an unpatched LaTeX2e can allow a
% single column figure to be placed prior to an earlier double column
% figure.
% Be aware that LaTeX2e kernels dated 2015 and later have fixltx2e.sty's
% corrections already built into the system in which case a warning will
% be issued if an attempt is made to load fixltx2e.sty as it is no longer
% needed.
% The latest version and documentation can be found at:
% http://www.ctan.org/pkg/fixltx2e


%\usepackage{stfloats}
% stfloats.sty was written by Sigitas Tolusis. This package gives LaTeX2e
% the ability to do double column floats at the bottom of the page as well
% as the top. (e.g., "\begin{figure*}[!b]" is not normally possible in
% LaTeX2e). It also provides a command:
%\fnbelowfloat
% to enable the placement of footnotes below bottom floats (the standard
% LaTeX2e kernel puts them above bottom floats). This is an invasive package
% which rewrites many portions of the LaTeX2e float routines. It may not work
% with other packages that modify the LaTeX2e float routines. The latest
% version and documentation can be obtained at:
% http://www.ctan.org/pkg/stfloats
% Do not use the stfloats baselinefloat ability as the IEEE does not allow
% \baselineskip to stretch. Authors submitting work to the IEEE should note
% that the IEEE rarely uses double column equations and that authors should try
% to avoid such use. Do not be tempted to use the cuted.sty or midfloat.sty
% packages (also by Sigitas Tolusis) as the IEEE does not format its papers in
% such ways.
% Do not attempt to use stfloats with fixltx2e as they are incompatible.
% Instead, use Morten Hogholm'a dblfloatfix which combines the features
% of both fixltx2e and stfloats:
%
% \usepackage{dblfloatfix}
% The latest version can be found at:
% http://www.ctan.org/pkg/dblfloatfix




% *** PDF, URL AND HYPERLINK PACKAGES ***
%
\usepackage{url}
% url.sty was written by Donald Arseneau. It provides better support for
% handling and breaking URLs. url.sty is already installed on most LaTeX
% systems. The latest version and documentation can be obtained at:
% http://www.ctan.org/pkg/url
% Basically, \url{my_url_here}.




% *** Do not adjust lengths that control margins, column widths, etc. ***
% *** Do not use packages that alter fonts (such as pslatex).         ***
% There should be no need to do such things with IEEEtran.cls V1.6 and later.
% (Unless specifically asked to do so by the journal or conference you plan
% to submit to, of course. )


% correct bad hyphenation here
\hyphenation{op-tical net-works semi-conduc-tor}


\begin{document}
%
% paper title
% Titles are generally capitalized except for words such as a, an, and, as,
% at, but, by, for, in, nor, of, on, or, the, to and up, which are usually
% not capitalized unless they are the first or last word of the title.
% Linebreaks \\ can be used within to get better formatting as desired.
% Do not put math or special symbols in the title.
\title{Smart Contract Oracles}


% author names and affiliations
% use a multiple column layout for up to three different
% affiliations
\author{Nina Minnich and Karoline Rabe}

% conference papers do not typically use \thanks and this command
% is locked out in conference mode. If really needed, such as for
% the acknowledgment of grants, issue a \IEEEoverridecommandlockouts
% after \documentclass

% for over three affiliations, or if they all won't fit within the width
% of the page, use this alternative format:
% 
%\author{\IEEEauthorblockN{Michael Shell\IEEEauthorrefmark{1},
%Homer Simpson\IEEEauthorrefmark{2},
%James Kirk\IEEEauthorrefmark{3}, 
%Montgomery Scott\IEEEauthorrefmark{3} and
%Eldon Tyrell\IEEEauthorrefmark{4}}
%\IEEEauthorblockA{\IEEEauthorrefmark{1}School of Electrical and Computer Engineering\\
%Georgia Institute of Technology,
%Atlanta, Georgia 30332--0250\\ Email: see http://www.michaelshell.org/contact.html}
%\IEEEauthorblockA{\IEEEauthorrefmark{2}Twentieth Century Fox, Springfield, USA\\
%Email: homer@thesimpsons.com}
%\IEEEauthorblockA{\IEEEauthorrefmark{3}Starfleet Academy, San Francisco, California 96678-2391\\
%Telephone: (800) 555--1212, Fax: (888) 555--1212}
%\IEEEauthorblockA{\IEEEauthorrefmark{4}Tyrell Inc., 123 Replicant Street, Los Angeles, California 90210--4321}}




% use for special paper notices
%\IEEEspecialpapernotice{(Invited Paper)}




% make the title area
\maketitle

% As a general rule, do not put math, special symbols or citations
% in the abstract
\begin{abstract}
The abstract goes here.
\end{abstract}

% no keywords




% For peer review papers, you can put extra information on the cover
% page as needed:
% \ifCLASSOPTIONpeerreview
% \begin{center} \bfseries EDICS Category: 3-BBND \end{center}
% \fi
%
% For peerreview papers, this IEEEtran command inserts a page break and
% creates the second title. It will be ignored for other modes.
\IEEEpeerreviewmaketitle



\section{Introduction}
%vielleicht etwas aus dem Ethereum Whitepaper - history


% no \IEEEPARstart
%This demo file is intended to serve as a ``starter file''
%for IEEE conference papers produced under \LaTeX\ using
%IEEEtran.cls version 1.8b and later.
% You must have at least 2 lines in the paragraph with the drop letter
% (should never be an issue)
%I wish you the best of success.

%\hfill mds
 
%\hfill August 26, 2015

\section{Methodology}

\section{Smart Contracts}
\subsection{Priciples and Applications}

Smart Contracts are scripts, which translate contract clauses into software code and execute them autonomously. Once implemented, these scripts cannot be manipulated, ensuring that the self-enforcement process will proceed according to the predefined rules in a deterministic way. Smart Contracts can ensure an efficient contracting process and reduce transactions costs by replacing the trusted third party, like courts or a notary \cite{Meitinger2017} \cite{Spancken2016}.\par  
 
Smart Contracts can be used for many different applications. An easy example would be a website which is sold from one contractual party to another. As soon as the purchaser pays price, which was stored in the Smart Contract, it will automatically transfer the property rights of the webiste. If physical objects are included, even a car rental service can be controlled by a Smart Contract. The script would watch if the payments occur in the agreed period and in case the borrower is overdue, the code will block the electronic car key. \cite{Jung2017} \cite{Lee2016}
\subsection{Blockchain}
The blockchain was first introduced by Satoshi Nakamoto with the cryptocurrency Bitcoin. The motivation behind Bitcoin was the need of an electronic payment system which is not based on trust but on cryptographic proof. This electronic payment system should enable transactions between two parties without any intermediate trusted third party.\cite{Nakamoto2008}
\subsubsection{Network, Blocks and Mining}
A blockchain is a distributed data structure shared by the nodes of the underlying peer-to-peer network. The name consisting of block and chain characterizes the structure and functionality. Blocks are identified by a cryptographic hash and each block has a reference to the block before via the hash. These hashes create a link between the blocks, thus a chain is built up. For of a blockchain there is no central authority needed. It is a decentralized peer-to-peer network without a trusted intermediary. Instead, a blockchain relies on cryptography. Each node has a pair of keys: a private and a public key. The private key is for signing the node's own transactions and with the public key the node can be addressed in the network. The asymmetric cryptography enables authentication, integrity, and nonrepudiation.\cite{Christidis2016} \par
Blockchains enable trustless networks. Transactions between nodes can be executed without one node trusting the other node. A blockchain network is formed by nodes operating on the same blockchain where each node holds a copy. The nodes interfere with the blockchain via their set of private/public key.\cite{Christidis2016} \par
Senders create and sign their transaction (e.g. transmission of crypto currencies or registration of a document) and they get sent to the client's neighbor peers who check if the received transaction is valid. Under which conditions a transaction is seen valid depends on certain rules. Every blockchain network has to set up rules for the decision of validity of transactions. Then the transcation is broadcasted to the whole blockchain network. The process of ordering and packing collected and validated transactions in a certain time interval into timestamped blocks is called mining. As a last step the mining node broadcasts the block to the network again. The conditions and details regarding the mining note depend on the consensus mechanism (see next subsection). The nodes of the network add the new block to their chain and execute the transaction if the block contains valid transactions and refers to the correct previous block in the chain via the hash value.\cite{Christidis2016}\cite{Prinz2017}\par
Different chains cannot communicate with each other, they cannot get external data or execute transactions on their own.\cite{Wang2017}\par
The first generation of blockchain did not support programmable transactions much. It only was a public ledger for monetary transactions with the typical application cryptocurrency. But with the second generation which represents a generally programmable infrastructure Smart Contracts were introduced.\cite{Xu2016} Other applications of blockchain which came up after the initial application of the cryptocurrency Bitcoin are for example colored coins, which make it possible to create own digital currencies, smart property, the ownership of physical devices, Namecoin, a decentralized name registration database, financial derivates, peer-to-peer gambling, smart contracts, etc.\cite{Buterin2014}

\subsubsection{Consensus Mechanism}
The content of the Blockchain is replicated in the decentralized peer to peer network, therefore all nodes in the network have to find a consensus regarding updates to ensure that only one network state is recognized as "the truth" by all parties. Especially in trustless networks like the Blockchain, special mechanisms have to be found for this purpose. \cite{Dinh?}\par 
\textit{Proof of Work} was the consensus mechanism used for Bitcoin. It means that some nodes in the network, the so called miners, have to solve cryptographical puzzles based on hash functions. There are CPU and memory based PoW variants but both have in common, that the solving time can be adapted by choosing the difficulty of the puzzle. No matter if the miners have to spend CPU cycles or wait for memory queries, PoW consumes always a decent amount of ressources, which is the main drawback of this consensus mechanism. \cite{Dinh?} \cite{Golze2009}
\textit{Proof of Stake} on the other hand is much less expensive than PoW regarding energy consumption because the puzzle's difficulty is adapted on the miner's stake in the network. The stake measures the participation in the Blockchain network and is mostly expressed by the amount of coins, that a miner holds in the respective cryprocurrency. Alternatively, there is also \textit{Proof of Burn}, where miners have to burn coins in order to mine new blocks and add new content to the Blockchain. Burning means sending money to an address where it cannot be spent. \cite{Dinh?} \par
The above-mentioned consensus mechanisms are suitable for public Blockchains, but there are also other architectures, which are based on private Blockchains or decentralized networks with some trusted parties. Especially for the last-mentioned, easier consensus mechanisms can be implemented, like \textit{Proof of Authority}, where the trusted parties update the Blockchain alternatly. In private Blockchains, where nodes are authenticated and a minimum of trust is assumed, communication based protocols like PBFT can be used, where all participants in the network have a right to vote and the consensus is reached through multiple voting rounds. \cite{Dinh?} \par 

\subsection{Technical Structure}

\subsubsection{Smart Contract Platforms}
%Ethereum --> würde ich als erstes machen, weil es die bekannteste Plattform ist.
\textit{Ethereum} is the most popular platform for smart contracts on the basis of blockchain. \cite{Meitinger2017} It has a public blockchain and its own currency called Ether (ETH). The consensus mechanism Ethereum uses is Proof of Work. In Ethereum accounts can be created which manage the virtual currency Ether and the smart contracts. \cite{Bartoletti2017} These accounts store the nonce (a counter so that every transaction is processed once), the ether balance, if existing the contract code and the account's storage. There are two types of accounts: Externally owned and contract accounts. The externally owned accounts are controlled by private keys and they can send messages by creating and signing a transaction. Contract accounts are controlled by their code and they run the code when receiving a message. Then a contract account can read and write to storage, send messages and create contracts. In Ethereum transactions are signed data packages which store messages to be sent by externally owned accounts. In contrast to the Bitcoin blockchain architecture, Ethereum blocks store a copy of the transaction list and the most recent state. \cite{Buterin2014}  \par
\textit{Bitcoin} was the first well-known decentralised cryptocurrency and the platform was originally created only for transferring transactions. With the rise of Ethereum and decentalized applications, developers started to use Bitcoin to execute protocols, which represent a limited form of Smart Contracts. Bitcoin's platform is suitable for this purpose because it uses its own public Blockchain to keep track of all transactions and Proof of Work as consensus mechanism, which creates the transparency and immutability that in necessary dealing with Smart Contracts. \cite{Bartoletti2017} \par 
\textit{Monax} in contrast is a platform without own currency, but which supports executing Ethereum Smart Contracts. Private Blockchains can be created, where the user defines his preferred authorization policy. Due to the lack of publicity in this platform, no intensive computation is needed for reaching consensus. Instead there are voting rounds, where a node can propose a block and if it reaches 2/3 of the of the vote, this block is added to the chain. \cite{Bartoletti2017} \par 
\textit{Lisk} has a public Blockchain with a currency like Bitcoin and Ethereum, but, instead of executing all applications on one main chain, Lisk provides separate Blockchains for every Smart Contract. The consensus mechanism in this platform is a delegated Proof of Stake, where the nodes can elect delegates, who are allowed to create blocks. The contract owner can decide which nodes should participate on the election process in the consensus mechanism of his customized subchain. \cite{Bartoletti2017} \par 
\textit{Hyperledger} was created to achieve a distributed ledger framework for industry applications and business transactions by the Linux Foundation. The platform for executing Smart Contracts is called Hyperledger Fabric and contains a permissioned Blockchain with two types of peers. The validating peer takes part in the consensus mechanism (PBFT) and validates transactions. The non-validating peer is not allowed to execute transactions but it can verify them and connect clients to the validating peers.\cite{Cachin2016} \par 
\textit{Counterparty} has no own blockchain. Instead, it adds data to Bitcoin transactions. Counterparty nodes can interpret the data and the Bitcoin nodes ignore this additional data. In contrast to other platforms no consensus mechanism is used. Counterparty has its own currency but there are no transaction fees but the paid fees are destroyed and the nodes are rewarded by the inflation (proof of burn). Smart contracts can be written in the same languages like in Ethereum. \cite{Bartoletti2017} \par 
\textit{Stellar} has a public blockchain and its own currency. Stellar does not have a certain programming language to create smart contracts but basic smart contracts can be implemented. To create more complex contracts transaction chaining and multi signature accounts are used. The application of the smart contracts go beyond handling of payments. An example is creating new accounts like special accounts called multisignature which can have more than one owner. The consensus mechanism based on the federated Byzantine agreement works as follows: nodes agree on transactions if the nodes in their neighborhood also agree.\cite{Bartoletti2017} \par 
%müssen wir zu dem consensus mechanism auch etwas im consensus mechanism Kapitel schreiben?
\textit{Tezos} is a generic and self-amending crypto-ledger and it can instantiate any blockchain based protocol. It starts with a seed protocol and then stakeholders can approve ideas for improvement. Tezos was developed because of the following issues of Bitcoin: The inability to innovate dynamically, cost and centralization problems of Proof of Work, the limited language to write smart contracts and security problems. In Tezos there is a limited maximum number of steps that a program may run for one transaction (transacitons per block and computation steps per block). Tezos supports Turing complete smart contracts and is implemented in the programming language OCaml. The seed protocol has a Proof of Stake mechanism.  \cite{Goodman2014}\cite{Goodman2014a} \par 

\subsubsection{Programming Languages}
The programming languages, that can be used to write Smart Contracts on the Blockchain are as various as the underlying platforms. The Ethereum network provides a runtime environment for Smart Contracts, the Ethereum virtual machine (EVM), which has a fixed word size of 32 bits and is untyped. The computation takes place on the stack with a maximum size of 1024 elements consisting of words of 256 bits. Entries accessed from the top end of the stack can be stored in the memory of the EVM. The EVM supports different types of memory. The persistent one for each account is called storage and is implemented as a key-value store, which is very costly to read or modify. The other type is called memory and created by the contract for each new message call. The memory supports reading limited to a width of 256 bits as well as writing of either 8 bits or 256 bits. \cite{Solidity2017} \par 
The original programming language for the EVM is a stack-based \textit{bytecode language}, which is Turing complete and supports a minimal instruction set consisting of arithmetic and logical operations as well as conditional and unconditional jumps. Useful operations for Smart Contracts are message calls, which are similar to transactions because they consist of source, target, payload as well as return data and can be used to call other contracts or send Ether to an account. Message calls are essential for data exchange between contracts. The called contract can access the call payload for retrieving transmitted data and after the execution return data via the call. With the feature delegatecall, a contract can even load code dynamically from another address at runtime. Besides, the EVM bytecode supports create calls to establish a new contract on the blockchain and the seld-destruct operation, which is the only posibility to remove contract code from the Blockchain. \cite{Bartoletti2017} \cite{McAdams2017} \cite{Solidity2017}\par 
Another original Ethereum smart contract programming language is \textit{Lisp Like Language - LLL}. LLL is a low level language similar to Assembler. With LLL the programmer writes EVM code without having to deal with stack management and jump management. The code in LLL consists of expressions to be evaluated instead of executable instructions. These LLL expressions can be integers, strings, atoms or an evaluated expression. Values of evaluated expressions are then put on top of the EVM stack. Almost all valid EVM opcodes are automatically valid LLL operations and in addition to the EVM opcodes there are some more operators. Furthermore amongst others there are LLL macros which help writing LLL code efficiently, the include expression to insert external expressions and control structures (seq, if, while, etc.). \cite{Edgington2017} \par 
Smart Contracts in Ethereum can also be written in high level languages, which are compiled to the Ethereum bytecode and later executed in the EVM. Solidity and Serpent are the most popular languages for Ethereum and therefore presented in the following. \textit{Solidity} was originally influenced by C++, Java, Python and JavaScript but in contrast to these existing multi-purpose languages, it was explicetly developed for Blockchain applications and is thus contract-oriented. Smart Contracts in Solidity can be compared to classes in object-oriented languages. Contracts consist of state variables, functions and can inherit from other contracts. Besides the trivial data types (integer, boolean, string, etc), Solidity supports contract specific variables. The type address can store a 20 byte value, which is the address size in the Ethereum network, and Mappings hold key value pairs and can be compared to hash tables. Events are special inheritable contracts members, which, when called, store the event data in a data structure on the Blockchain where it is no longer accessible. \cite{Solidity2017} \cite{McAdams2017} \par 
\textit{Serpent} on the other hand is designed similar to Python with the goal to be very clean and simple. Serpent combines the advantages of the efficient low level languages with the user-friendly high level options. Serpent is similar to Solidity in the way that it also supports all trivial data types. Serpent provides special contract-oriented variables like "msg.sender" to store the sending person's address or "block.timestamp" to store the time when the current block was created. Serpent allows to create persistent data structures by using the the data declaration, which will exist throughout the contract execution. \cite{McAdams2017} \cite{Arnett2015} \par 
The language used for programming in Bitcoin is called \textit{Script}. Script is a bytecode stack-based, not Turing complete language which is designed purposely, so it is assured that its execution terminates. Script provides more than hundred primitives but there are no variables, arrays, functions and looping or recursion. In contrast to other stack-based languages Script provides the possibility of random access instead of access to the top elements on the stack only. The instructions OP\_PICK and OP\_ROLL copy/move an element to the top of the stack and make reading access to previously calculated elements possible. Scripts written in the Bitcoin scripting language contain a sequence of instructions which are executed in this order without jumps. The limitations of Script are on the size of blocks, the length of the scripts and the number of opcodes. \cite{McAdams2017} \par 
Bitcoin and Ethereum provide a virtual machine (low level computation model) and higher level languages can be compiled into this computation model. In other systems there is a higher-level API in a general purpose language like \textit{JavaScript}. The syntax of Solidity is similar to the syntax of JavaScript. \cite{McAdams2017} The data types of JavaScript are string, number, boolean, array and object. \cite{McPeak2015} \par 

\subsection{Data Feeds and Processing}

\section{Oracles}
\subsection{Definition}
\subsection{Implementation}
\subsubsection{Decentralized Oracles}
\subsubsection{Untampered Data Sources}
\subsubsection{Trusted Hardware}
Another solution for gathering data for smart contracts is trusted hardware. The oracle based on trusted hardware connects a blockchain with trusted hardware. There is a connection between a smart contract and a off-chain trusted computer environment. A trusted hardware component is one possible answer to the general issue of smart contracts: the lack of a trustworthy data feed.\par 
An example for this approach is Town Crier. Town Crier connects smart contracts with existing websites which already are commonly trusted for non-blockchain applications. In more detail Twon Crier creates a connection between HTTPS-enabled websites and the Ethereum blockchain and provides an authenticated data feed respectively oracle. It gathers data from the websites and passes it on to the inquiring(??) smart contract in form of so called datagrams.\par 
Town Crier uses Intel’s SGX - Software Guard Extension and a smart contract front end. With SGX it is possible to execute processes in an isolated enclave which protects the process from manipulation by hardware attacks and software on the same host, even the operating system. Town Crier consists of three main components: The TC Contract C$_{TC}$ (on the blockchain), the Enclave and the Relay (on the Town Crier server). The C$_{TC}$ represents a simple API to a requesting smart contract which wants to use Town Crier's services of getting data from a website. It gets the requests from requester/relying contract, returns the datagrams from Town Crier and manages TC's monetary resources.
Town Crier is especially developed for the platform Ethereum.
In addition to normal requests which are visible on the blockchain Town Crier also supports private and custom requests.
% An example of a floating figure using the graphicx package.
% Note that \label must occur AFTER (or within) \caption.
% For figures, \caption should occur after the \includegraphics.
% Note that IEEEtran v1.7 and later has special internal code that
% is designed to preserve the operation of \label within \caption
% even when the captionsoff option is in effect. However, because
% of issues like this, it may be the safest practice to put all your
% \label just after \caption rather than within \caption{}.
%
% Reminder: the "draftcls" or "draftclsnofoot", not "draft", class
% option should be used if it is desired that the figures are to be
% displayed while in draft mode.
%
%\begin{figure}[!t]
%\centering
%\includegraphics[width=2.5in]{myfigure}
% where an .eps filename suffix will be assumed under latex, 
% and a .pdf suffix will be assumed for pdflatex; or what has been declared
% via \DeclareGraphicsExtensions.
%\caption{Simulation results for the network.}
%\label{fig_sim}
%\end{figure}

% Note that the IEEE typically puts floats only at the top, even when this
% results in a large percentage of a column being occupied by floats.


% An example of a double column floating figure using two subfigures.
% (The subfig.sty package must be loaded for this to work.)
% The subfigure \label commands are set within each subfloat command,
% and the \label for the overall figure must come after \caption.
% \hfil is used as a separator to get equal spacing.
% Watch out that the combined width of all the subfigures on a 
% line do not exceed the text width or a line break will occur.
%
%\begin{figure*}[!t]
%\centering
%\subfloat[Case I]{\includegraphics[width=2.5in]{box}%
%\label{fig_first_case}}
%\hfil
%\subfloat[Case II]{\includegraphics[width=2.5in]{box}%
%\label{fig_second_case}}
%\caption{Simulation results for the network.}
%\label{fig_sim}
%\end{figure*}
%
% Note that often IEEE papers with subfigures do not employ subfigure
% captions (using the optional argument to \subfloat[]), but instead will
% reference/describe all of them (a), (b), etc., within the main caption.
% Be aware that for subfig.sty to generate the (a), (b), etc., subfigure
% labels, the optional argument to \subfloat must be present. If a
% subcaption is not desired, just leave its contents blank,
% e.g., \subfloat[].


% An example of a floating table. Note that, for IEEE style tables, the
% \caption command should come BEFORE the table and, given that table
% captions serve much like titles, are usually capitalized except for words
% such as a, an, and, as, at, but, by, for, in, nor, of, on, or, the, to
% and up, which are usually not capitalized unless they are the first or
% last word of the caption. Table text will default to \footnotesize as
% the IEEE normally uses this smaller font for tables.
% The \label must come after \caption as always.
%
%\begin{table}[!t]
%% increase table row spacing, adjust to taste
%\renewcommand{\arraystretch}{1.3}
% if using array.sty, it might be a good idea to tweak the value of
% \extrarowheight as needed to properly center the text within the cells
%\caption{An Example of a Table}
%\label{table_example}
%\centering
%% Some packages, such as MDW tools, offer better commands for making tables
%% than the plain LaTeX2e tabular which is used here.
%\begin{tabular}{|c||c|}
%\hline
%One & Two\\
%\hline
%Three & Four\\
%\hline
%\end{tabular}
%\end{table}


% Note that the IEEE does not put floats in the very first column
% - or typically anywhere on the first page for that matter. Also,
% in-text middle ("here") positioning is typically not used, but it
% is allowed and encouraged for Computer Society conferences (but
% not Computer Society journals). Most IEEE journals/conferences use
% top floats exclusively. 
% Note that, LaTeX2e, unlike IEEE journals/conferences, places
% footnotes above bottom floats. This can be corrected via the
% \fnbelowfloat command of the stfloats package.




\section{Conclusion}
The conclusion goes here.




% conference papers do not normally have an appendix


% use section* for acknowledgment
\section*{Acknowledgment}


The authors would like to thank...





% trigger a \newpage just before the given reference
% number - used to balance the columns on the last page
% adjust value as needed - may need to be readjusted if
% the document is modified later
%\IEEEtriggeratref{8}
% The "triggered" command can be changed if desired:
%\IEEEtriggercmd{\enlargethispage{-5in}}

% references section

% can use a bibliography generated by BibTeX as a .bbl file
% BibTeX documentation can be easily obtained at:
% http://mirror.ctan.org/biblio/bibtex/contrib/doc/
% The IEEEtran BibTeX style support page is at:
% http://www.michaelshell.org/tex/ieeetran/bibtex/
\bibliographystyle{IEEEtran}
% argument is your BibTeX string definitions and bibliography database(s)
\bibliography{Literatur}
%
% <OR> manually copy in the resultant .bbl file
% set second argument of \begin to the number of references
% (used to reserve space for the reference number labels box)





% that's all folks
\end{document}


